\documentclass{article}

\usepackage{amsmath}
\usepackage{amssymb}
\usepackage{xcolor}
\usepackage{graphicx}
\usepackage[alsoload=binary]{siunitx}
\usepackage{booktabs}
\usepackage{multirow}
\usepackage{fancyvrb}
\usepackage[section]{placeins}
\usepackage{flafter}
\usepackage{bm}
\usepackage{url}
\usepackage{hyperref}

\renewcommand\vec[1]{\bm{#1}}

% a bit more compact
\let\l\left
\let\r\right

% no section numbers
\setcounter{secnumdepth}{-2}

% from amssymb.sty
\let\emtpyset\varnothing

% leave notes to yourself
\newcommand\todo[1]{\textcolor{red}\textsc{todo}: #1}

% write in code
\DefineShortVerb{\|}

\title{Parallel Implementation of Phase Field Modeling for Solidification}
\author{Sam Britt}
\date{Oct. 12, 2012}

\pagestyle{empty}

\begin{document}
  \maketitle
  \thispagestyle{empty}

  \section{Background}
  \label{sec:background}

  Solidification continues to be of great importance in materials
  science and materials processing. Engineering alloys cast from the
  melt undergo extreme microstructural changes during the
  solidification process, and the resulting morphological and
  topological characteristics direct subsequent work on the material,
  which ultimately determines its properties. Small variations in
  composition and cooling rates can have profound effects on the
  resulting solidified microstructure, for example, the grain size
  distribution, the appearance of columnar grains, or the growth of
  dendrites. The ability to efficiently and accurately predict
  solidification microstructures helps engineers increase the rate of
  development of new materials and materials processing techniques.

  The phase field method has been successfully applied in studying
  various types of microstructure evolution, including solidification,
  solid-state phase transformations, grain growth, and dislocation
  dynamics \cite{Chen2002}. The method describes a microstructure via
  a set of ``order parameters" $\phi_i(\vec{x}, t)$, which must sum to
  $1$, and together represent the ``phase state" at time $t$ at
  location $\vec x$ in the microstructure \cite{Ode2001}.  For
  example, if there are three phases in the microstructure, and at
  location $\vec{x}$, $\phi_1 = \phi_2 = 0$ and $\phi_3 = 1$, then the
  microstructure at $\vec x$ is completely phase three. If, at another
  location $\vec x'$, $\phi_1 = 0$, $\phi_2 = 0.25$, and $\phi_3 =
  0.75$, then $\vec x'$ would be located at an interface between
  phases two and three. Sharp interphase boundaries are avoided by
  letting the $\phi_i$ functions vary continuously, and interface
  movement can be modeled without tracking interfaces
  explicitly---interface velocity simply arises due to the $\phi_i(\vec
  x)$ parameters changing through time.

  For basic solidification problems, there are only two phases
  present: solid and liquid. Since in this case $\phi_1 = 1 - \phi_2$
  always, the equations are usually written in terms of a single order
  parameter $\phi$, representing the solid phase.

  The time change of $\phi$ is modeled as proportional to the
  variation in a free energy functional,
  \begin{equation}
    \frac{\partial \phi(\vec{x}, t)}{\partial t} =
    M \frac{\delta F}{\delta \phi(\vec{x}, t)},
    \label{eq:dphi-dt}
  \end{equation}
  where $M$ is related to the mobility of the phase, and $F$ is an
  appropriately chosen free energy; for solidification the Helmholtz
  free energy is often used. The variation of $F$ provides the driving
  force for the interface movement. Increasingly complex expressions
  for $F$ can capture bulk free energy, interfacial free energy, and
  free energy contributions from longer range interactions, such as
  elastic (mechanical) and electrostatic interactions between phases
  \cite{Chen2002}. Diffusion equations and the effects of temperature
  gradients can be solved simultaneously with Eqn.~\eqref{eq:dphi-dt}.
  Given initial and boundary conditions, Eqn.~\eqref{eq:dphi-dt} can
  be solved for all $\vec x$ at each timestep to track the growth and
  shape of the solidifying precipitates.

  \section{Project Description}
  \label{sec:project_description}

  The goal of this project is to implement a parallelized version of
  the phase field model. Since the bulk of the effort will be focused
  on building the infrastructure, the problem space will be limited to
  a basic solidification problem in, e.g., a Cu-Al system. Since the
  project will focus on such a well-studied system, there should be
  sufficient microstructural data and materials parameters available
  in the literature. However, the code should be sufficiently modular
  so that it can be later extended; for example, more advanced
  treatments of the free energy functional could be plugged in, or
  more phases could be added.

  Traditionally, Eqn.~\eqref{eq:dphi-dt} has been solved using finite
  difference methods, though other methods are possible, e.g., Fourier
  transform methods \cite{Chen2002}. For this project, I intend to use
  the finite difference method. Since Eqn.~\eqref{eq:dphi-dt} is over
  a spatial domain, it should parallelize quite naturally over a
  computer cluster, where each node communicates at its boundary. One
  advanced technique that I would like to pursue, if time permits, is
  an adaptive mesh over the spatial domain. Since the most interesting
  parts of the simulation are the interphase regions; that is,
  where $\phi$ transitions from $0$ to $1$, it would be ideal if $\vec
  x$ was discretized such that it had the finest mesh size in those
  locations. These areas of finer mesh would travel with the
  interfaces.

  Final deliverables include the parallelized phase field method code,
  any post-processing or visualization code, and the final report.

  \bibliographystyle{elsarticle-num}
  \bibliography{library}
\end{document}
