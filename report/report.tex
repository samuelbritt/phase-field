\documentclass{article}

\usepackage{mathpazo}

\usepackage{amsmath}
\usepackage{amssymb}
\usepackage{xcolor}
\usepackage{graphicx}
\usepackage[alsoload=binary]{siunitx}
\usepackage{booktabs}
\usepackage{multirow}
\usepackage{fancyvrb}
\usepackage[section]{placeins}
\usepackage{flafter}
\usepackage{url}
\usepackage{hyperref}

% make plots
\usepackage{pgfplots}

% a bit more compact
\renewcommand\l{\mathopen{}\left}
\renewcommand\r{\right}

% no section numbers
\setcounter{secnumdepth}{-2}

% leave notes to yourself
\newcommand\todo[1]{\textcolor{red}{\textsc{todo}: #1}}

\newcommand\abs[1]{\l\vert #1 \r\vert}
\newcommand\grad[1]{\nabla #1}
\newcommand\lap[1]{\nabla^2 #1}
\let\epsilon\varepsilon

\usepackage{bm}
\renewcommand\vec[1]{\bm{#1}}
\newcommand\unit[1]{\hat{\vec{#1}}}

\begin{document}

\begin{equation*}
  F = \int_V \l[
    f(\phi, T)
    +
    \frac{\epsilon^2}{2} \abs{\grad\phi}^2
  \r]
  dV
\end{equation*}

\begin{align}
  \frac{\partial \phi}{\partial t}
  &= M
  \l[
    \epsilon^2 \lap{\phi}
    -\frac{\partial f(\phi, T)}{\partial \phi}
  \r]
  \label{eq:allen-cahn}
  \\
  \frac{\partial u}{\partial t}
  &= D \lap u +
  \frac{1}{2} \frac{\partial \phi}{\partial t}
  \label{eq:heat-eqn}
  \\
  u &\equiv \frac{T - T_m}{L / c_P}
  \notag
\end{align}

\begin{align}
  f(\phi, T) &= \l[ 1 - p(\phi) \r] f_S(T) + p(\phi) f_L(T) + Q
  g(\phi) \notag \\
             &= f_S(T) + p(\phi) \l[f_L(T) - f_S(T)\r]+ Q g(\phi)
                \label{eq:bulk_energy} \\
     p(\phi) &= \phi^3 \l( 6 \phi^2 - 15\phi + 10 \r) \notag \\
     g(\phi) &= \phi^2 \l( 1 - \phi \r)^2 \notag
\end{align}

\begin{tikzpicture}
  \begin{axis}
    [
      scale only axis,
      domain=-0.1:1.1,
      ymin=0,
      ymax=1,
      no markers,
      xlabel=$\phi$,
      ylabel=$10 g(\phi)$ or $p(\phi)$,
      samples=100,
      legend style={
        at={(0.03,0.97)},
        anchor=north west
      }
    ]
    \addplot expression {10*x^2 * (1 - x)^2};             % 10 g(phi)
    \addplot expression {x^3 * (6* x^2 - 15 * x + 10)};   % p(phi)
    \legend{$10 g(\phi)$,$p(\phi)$}
  \end{axis}
\end{tikzpicture}

Simplifications on $f(\phi, T)$:
\begin{enumerate}
  \item Set solid as standard state, so that, for all $T$:
    \begin{equation*}
      f_S(T) \equiv 0.
    \end{equation*}
  \item Use the common approximation for alloys at $T \approx T_m$:
    \begin{align*}
      \Delta f_{\text{melt}} = f_L(T) - f_S(T)
      &\approx \frac{L \l( T_m - T \r)}{T_m} \\
      &= - \frac{L^2}{c_P T_m} u \\
      &= - \kappa u
    \end{align*}
    where $\kappa = L^2/ \l( c_P T_m \r)$.
\end{enumerate}
Therefore, changing from $T$ to $u$ and substituting
into~\eqref{eq:bulk_energy}, we have
\begin{equation}
  \label{eq:bulk_energy_solved}
  f(\phi, u) = - \kappa u p(\phi) + Q g(\phi).
\end{equation}

In one dimension, the interface thickness $\delta$ is given by
\begin{equation*}
  \delta = \frac{\epsilon}{\sqrt{2 Q}}
\end{equation*}
and the surface free energy $\sigma$ is
\begin{equation*}
  \sigma = \frac{\epsilon \sqrt{Q}}{3 \sqrt{2}}.
\end{equation*}

To introduce anisotropy, let $\epsilon = \epsilon(\unit{n})$, where
$\unit{n}$ is the unit vector normal to the to the interface.
Following Baragard,
\begin{equation*}
  \epsilon^2\l( \unit n \r)
    = \epsilon_0^2
        \l( 1 - 3 \epsilon_c \r)
        \l[
          1  + \frac{4 \epsilon_c}{1 - 3 \epsilon_c}
          \l(
            n_x^4 + n_y^4 + n_z^4
          \r)
        \r]
\end{equation*}
where $\epsilon_c$ is a constant, $\epsilon$ reduces to $\epsilon_0$
in the isotropic case, and $\unit n$ is
taken as $- \grad \phi / \abs{ \grad \phi }$, so that
\begin{equation*}
  n_x^4 + n_y^4 + n_z^4 =
  \frac{
    \l( \partial \phi / \partial x \r)^4 +
    \l( \partial \phi / \partial y \r)^4 +
    \l( \partial \phi / \partial z \r)^4
  }{
    \abs{ \grad \phi }^4
  }.
\end{equation*}

\section{Numerical Analysis} 
Finding the derivative of Eqn.~\eqref{eq:bulk_energy_solved} and
substituting into~\eqref{eq:allen-cahn}, we have
\begin{equation*}
  \frac {\partial \phi }{ \partial t } =
  M \epsilon^2 \lap{\phi} +
  2 M \phi \l[ 
    15 \kappa u \phi \l( \phi - 1 \r)^2 -
    Q \l( 1 - \phi \r) \l( 1 - 2 \phi \r)
  \r]
\end{equation*}
which is to be integrated numerically, coupled with
Eqn.~\eqref{eq:heat-eqn}. Let $h\l( \phi, u \r)$ be the second term
on the right hand side of the above equation, so
\begin{equation*}
  \frac {\partial \phi }{ \partial t }
  = M \epsilon^2 \lap{\phi} + h\l( \phi, u \r)
\end{equation*}

\end{document}
